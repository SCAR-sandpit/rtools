\documentclass[]{article}
\usepackage{lmodern}
\usepackage{amssymb,amsmath}
\usepackage{ifxetex,ifluatex}
\usepackage{fixltx2e} % provides \textsubscript
\ifnum 0\ifxetex 1\fi\ifluatex 1\fi=0 % if pdftex
  \usepackage[T1]{fontenc}
  \usepackage[utf8]{inputenc}
\else % if luatex or xelatex
  \ifxetex
    \usepackage{mathspec}
  \else
    \usepackage{fontspec}
  \fi
  \defaultfontfeatures{Ligatures=TeX,Scale=MatchLowercase}
\fi
% use upquote if available, for straight quotes in verbatim environments
\IfFileExists{upquote.sty}{\usepackage{upquote}}{}
% use microtype if available
\IfFileExists{microtype.sty}{%
\usepackage{microtype}
\UseMicrotypeSet[protrusion]{basicmath} % disable protrusion for tt fonts
}{}
\usepackage[margin=1in]{geometry}
\usepackage{hyperref}
\hypersetup{unicode=true,
            pdftitle={RTools for Antarctica and the Southern Ocean Biodiversity},
            pdfauthor={Hsun-Yi Hsieh},
            pdfborder={0 0 0},
            breaklinks=true}
\urlstyle{same}  % don't use monospace font for urls
\usepackage{color}
\usepackage{fancyvrb}
\newcommand{\VerbBar}{|}
\newcommand{\VERB}{\Verb[commandchars=\\\{\}]}
\DefineVerbatimEnvironment{Highlighting}{Verbatim}{commandchars=\\\{\}}
% Add ',fontsize=\small' for more characters per line
\usepackage{framed}
\definecolor{shadecolor}{RGB}{248,248,248}
\newenvironment{Shaded}{\begin{snugshade}}{\end{snugshade}}
\newcommand{\KeywordTok}[1]{\textcolor[rgb]{0.13,0.29,0.53}{\textbf{#1}}}
\newcommand{\DataTypeTok}[1]{\textcolor[rgb]{0.13,0.29,0.53}{#1}}
\newcommand{\DecValTok}[1]{\textcolor[rgb]{0.00,0.00,0.81}{#1}}
\newcommand{\BaseNTok}[1]{\textcolor[rgb]{0.00,0.00,0.81}{#1}}
\newcommand{\FloatTok}[1]{\textcolor[rgb]{0.00,0.00,0.81}{#1}}
\newcommand{\ConstantTok}[1]{\textcolor[rgb]{0.00,0.00,0.00}{#1}}
\newcommand{\CharTok}[1]{\textcolor[rgb]{0.31,0.60,0.02}{#1}}
\newcommand{\SpecialCharTok}[1]{\textcolor[rgb]{0.00,0.00,0.00}{#1}}
\newcommand{\StringTok}[1]{\textcolor[rgb]{0.31,0.60,0.02}{#1}}
\newcommand{\VerbatimStringTok}[1]{\textcolor[rgb]{0.31,0.60,0.02}{#1}}
\newcommand{\SpecialStringTok}[1]{\textcolor[rgb]{0.31,0.60,0.02}{#1}}
\newcommand{\ImportTok}[1]{#1}
\newcommand{\CommentTok}[1]{\textcolor[rgb]{0.56,0.35,0.01}{\textit{#1}}}
\newcommand{\DocumentationTok}[1]{\textcolor[rgb]{0.56,0.35,0.01}{\textbf{\textit{#1}}}}
\newcommand{\AnnotationTok}[1]{\textcolor[rgb]{0.56,0.35,0.01}{\textbf{\textit{#1}}}}
\newcommand{\CommentVarTok}[1]{\textcolor[rgb]{0.56,0.35,0.01}{\textbf{\textit{#1}}}}
\newcommand{\OtherTok}[1]{\textcolor[rgb]{0.56,0.35,0.01}{#1}}
\newcommand{\FunctionTok}[1]{\textcolor[rgb]{0.00,0.00,0.00}{#1}}
\newcommand{\VariableTok}[1]{\textcolor[rgb]{0.00,0.00,0.00}{#1}}
\newcommand{\ControlFlowTok}[1]{\textcolor[rgb]{0.13,0.29,0.53}{\textbf{#1}}}
\newcommand{\OperatorTok}[1]{\textcolor[rgb]{0.81,0.36,0.00}{\textbf{#1}}}
\newcommand{\BuiltInTok}[1]{#1}
\newcommand{\ExtensionTok}[1]{#1}
\newcommand{\PreprocessorTok}[1]{\textcolor[rgb]{0.56,0.35,0.01}{\textit{#1}}}
\newcommand{\AttributeTok}[1]{\textcolor[rgb]{0.77,0.63,0.00}{#1}}
\newcommand{\RegionMarkerTok}[1]{#1}
\newcommand{\InformationTok}[1]{\textcolor[rgb]{0.56,0.35,0.01}{\textbf{\textit{#1}}}}
\newcommand{\WarningTok}[1]{\textcolor[rgb]{0.56,0.35,0.01}{\textbf{\textit{#1}}}}
\newcommand{\AlertTok}[1]{\textcolor[rgb]{0.94,0.16,0.16}{#1}}
\newcommand{\ErrorTok}[1]{\textcolor[rgb]{0.64,0.00,0.00}{\textbf{#1}}}
\newcommand{\NormalTok}[1]{#1}
\usepackage{graphicx,grffile}
\makeatletter
\def\maxwidth{\ifdim\Gin@nat@width>\linewidth\linewidth\else\Gin@nat@width\fi}
\def\maxheight{\ifdim\Gin@nat@height>\textheight\textheight\else\Gin@nat@height\fi}
\makeatother
% Scale images if necessary, so that they will not overflow the page
% margins by default, and it is still possible to overwrite the defaults
% using explicit options in \includegraphics[width, height, ...]{}
\setkeys{Gin}{width=\maxwidth,height=\maxheight,keepaspectratio}
\IfFileExists{parskip.sty}{%
\usepackage{parskip}
}{% else
\setlength{\parindent}{0pt}
\setlength{\parskip}{6pt plus 2pt minus 1pt}
}
\setlength{\emergencystretch}{3em}  % prevent overfull lines
\providecommand{\tightlist}{%
  \setlength{\itemsep}{0pt}\setlength{\parskip}{0pt}}
\setcounter{secnumdepth}{0}
% Redefines (sub)paragraphs to behave more like sections
\ifx\paragraph\undefined\else
\let\oldparagraph\paragraph
\renewcommand{\paragraph}[1]{\oldparagraph{#1}\mbox{}}
\fi
\ifx\subparagraph\undefined\else
\let\oldsubparagraph\subparagraph
\renewcommand{\subparagraph}[1]{\oldsubparagraph{#1}\mbox{}}
\fi

%%% Use protect on footnotes to avoid problems with footnotes in titles
\let\rmarkdownfootnote\footnote%
\def\footnote{\protect\rmarkdownfootnote}

%%% Change title format to be more compact
\usepackage{titling}

% Create subtitle command for use in maketitle
\newcommand{\subtitle}[1]{
  \posttitle{
    \begin{center}\large#1\end{center}
    }
}

\setlength{\droptitle}{-2em}
  \title{RTools for Antarctica and the Southern Ocean Biodiversity}
  \pretitle{\vspace{\droptitle}\centering\huge}
  \posttitle{\par}
  \author{Hsun-Yi Hsieh}
  \preauthor{\centering\large\emph}
  \postauthor{\par}
  \date{}
  \predate{}\postdate{}

\usepackage{xcolor}
\usepackage{framed}

\begin{document}
\maketitle

\subsection{Occurrence data retrieval}\label{occurrence-data-retrieval}

\href{https://www.gbif.org/}{The Global Biodiversity Information
Facility (GBIF)}, \href{http://www.iobis.org/}{the Ocean Biogeographic
Information System (OBIS)} and ANTABIF all provide species occurrence
data. As the GBIF harbors occurrences of both terrestrial and marine
species at the globa scale, the OBIS is the most comprehensive source of
marine species occurrences and ANTABIF aims to establish as an authority
of biodiversity data for Antarctica and the Southern Ocean.

The construction of the ANTABIF API is an ongoing project. This page
presents the retrieval, analysis and visualization of GBIF and OBIS
occurrence data. The following sections will provide a basic workflow
for data cleaning, wrangling, analysis and visualisation in R.

\subsubsection{GBIF}\label{gbif}

rgbif helps users retrieve data from the GBIF.

\begin{Shaded}
\begin{Highlighting}[]
\KeywordTok{writeLines}\NormalTok{(}\StringTok{"occ_search() is a function for searching GBIF occurrence data."}\NormalTok{)}
\end{Highlighting}
\end{Shaded}

\begin{verbatim}
## occ_search() is a function for searching GBIF occurrence data.
\end{verbatim}

\colorlet{shadecolor}{gray!10}

\color{red}

\begin{Shaded}
\begin{Highlighting}[]
\KeywordTok{writeLines}\NormalTok{(}\StringTok{"help"}\NormalTok{)}
\end{Highlighting}
\end{Shaded}

\begin{verbatim}
## help
\end{verbatim}

\section{search by scientificName.}\label{search-by-scientificname.}

occ\_search(scientificName = `Ursus americanus', limit = 0, return =
``meta'')

\section{search by dataset key.}\label{search-by-dataset-key.}

occ\_search(datasetKey=`7b5d6a48-f762-11e1-a439-00145eb45e9a',
return=`data', limit=20)

\section{Search on latitidue and
longitude}\label{search-on-latitidue-and-longitude}

occ\_search(search=``kingfisher'', decimalLatitude=50,
decimalLongitude=-10)

\begin{verbatim}
Alternatively, users can download data directly from GBIF. For instance, [Antarctic Plant Database]("https://www.gbif.org/dataset/82d9ff5c-f762-11e1-a439-00145eb45e9a") harbors more than 50,000 occurrences of over 40,000 plant specimens from Antarctica, the sub-Antarctic islands and surrounding continents. 

Users can also download and import the data directly in one step as follows.
\end{verbatim}

dd\_gbif \textless{}- occ\_download\_get(key =
``0000066-140928181241064'', overwrite = TRUE) \%\textgreater{}\%
occ\_download\_import(dd\_gbif\_download, na.strings = c(``'', NA))

\begin{verbatim}
### OBIS
<a href = "https://github.com/iobis/robis">robis</a> helps users to retrieve data from the OBIS.
\end{verbatim}

install.packages(``rgbif'') library(``rgbif'') occ\_get(key=c(101010,
240713150, 855998194), return=`data') occ\_get(key=c(855998194,
620594291, 766420684), fields=c(`scientificName', `decimalLatitude',
`basisOfRecord'), verbatim=TRUE)

\begin{verbatim}

## Data browsing, cleaning wrangling
### tidyverse
The <a href = "https://www.tidyverse.org/packages/"> **tidyverse**</a> is a collection of R packages designed for data science. The collection includes <a href = "http://dplyr.tidyverse.org/">**dplyr**</a> for data frame creation and manipulation, <a href = "http://readr.tidyverse.org/">**readr**</a> and <a href = "http://readxl.tidyverse.org/">**readxl**</a> for reading data from text or EXCEL files, <a href = "http://purrr.tidyverse.org/">**purrr**</a> for speeding up functional prorgramminga and <a href = "http://ggplot2.tidyverse.org/">**ggplot2**</a> for visualization. 

Users can install all the packages in the tidyverse by running 
\end{verbatim}

install.packages(``tidyverse'')

\begin{verbatim}


Alternatively, install one or more single package(s) by running, for instance,
\end{verbatim}

install.packages(``tidyr'') install.packages(``dplyr'')

\begin{verbatim}

Or download the development versions from GitHub
\end{verbatim}

devtools::install\_github(``tidyverse/tidyr'')
devtools::install\_github(``tidyverse/dplyr'') ```

\subsubsection{rrefine}\label{rrefine}

OpenRefine is an open source data cleaning software. rrefine allows
users to programmatically triger data transfer between R and
`OpenRefine'.

\subsubsection{obistools}\label{obistools}

\subsection{Taxonomy Tools}\label{taxonomy-tools}

\subsubsection{taxasize}\label{taxasize}

\subsubsection{biotaxa}\label{biotaxa}

biotaxa is a tool for the exploration and visualization of taxa
discovery.

To install the package,

To user \texttt{biotaxa}, the dataset should contain two columns of taxa
classifications (e.g.~kingdom, phylum, class, order, family, genus,
species or AphiaID) and taxa discovery year. Take a look of the example
dataset,

\begin{Shaded}
\begin{Highlighting}[]
\CommentTok{#head(data_m)}
\end{Highlighting}
\end{Shaded}

To visualize the accmulation curve of all genera belonging to Animalia,
use \texttt{taxaaccum()}.

\begin{Shaded}
\begin{Highlighting}[]
\CommentTok{#taxaaccum("Animalia", "Genus")}
\end{Highlighting}
\end{Shaded}

To list and rank all of the genera belonging to Family `Salpidae' fc
\textless{}- frequencyrank(``Salpidae'', ``Genus'') fc \textless{}-
fc{[}fc\$Genus != ``'', {]} \#fc

\subsection{Visualization}\label{visualization}

The
\href{https://github.com/AustralianAntarcticDataCentre/antanym-demo}{Github
page of Australian Antarctic Data Center} hosts nice source code for
polar stereographic projection.

To run the piece of code, first install the leaflet package of the
rstudio version.

\begin{verbatim}
devtools::install_github("rstudio/leaflet")
library(leaflet)
library(dplyr)
\end{verbatim}

\begin{verbatim}
library(antanym)
library(leaflet)
g <- an_read()

## find single name per feature, preferring United Kingdom
##  names where available, and only rows with valid locations
temp <- g %>% an_preferred("United Kingdom")
temp <- fc <- na.omit(fc)temp[!is.na(temp$longitude) & !is.na(temp$latitude),]

## replace NAs with empty strings in narrative
temp$narrative[is.na(temp$narrative)] <- ""

## formatted popup HTML
popup <- sprintf("<h1>%s</h1><p><strong>Country of origin:</strong> %s<br />
  <strong>Longitude:</strong> %g<br /><strong>Latitude:</strong> %g<br />
  <a href=\"https://data.aad.gov.au/aadc/gaz/scar/display_name.cfm?gaz_id=%d\">
    Link to SCAR gazetteer</a></p>",temp$place_name,temp$country_name,
  temp$longitude,temp$latitude,temp$gaz_id)

m <- leaflet() %>%
  addProviderTiles("Esri.WorldImagery") %>%
  addMarkers(lng = temp$longitude, lat = temp$latitude, group = "placenames",
    clusterOptions = markerClusterOptions(),popup = popup,
    label = temp$place_name)

startZoom <- 1

crsAntartica <-  leafletCRS(
  crsClass = 'L.Proj.CRS',
  code = 'EPSG:3031',
  proj4def = '+proj=stere +lat_0=-90 +lat_ts=-71 +lon_0=0 +k=1 +x_0=0 +y_0=0 +ellps=WGS84 +datum=WGS84 +units=m +no_defs',
  resolutions = c(8192, 4096, 2048, 1024, 512, 256),
  origin = c(-4194304, 4194304),
  bounds =  list( c(-4194304, -4194304), c(4194304, 4194304) )
)

mps <- leaflet(options = leafletOptions(crs = crsAntartica, minZoom = 0, worldCopyJump = FALSE)) %>%
    setView(0, -90, startZoom) %>%
    addCircleMarkers(lng = temp$longitude, lat = temp$latitude, group = "placenames",
                     popup = popup, label = temp$place_name,
                     fillOpacity = 0.5, radius = 8, stroke = FALSE, color = "#000",
                     labelOptions = labelOptions(textOnly = FALSE)) %>%
    addWMSTiles(baseUrl = "https://maps.environments.aq/mapcache/antarc/?",
                layers = "antarc_ramp_bath_shade_mask",
                options = WMSTileOptions(format = "image/png", transparent = TRUE),
                attribution = "Background imagery courtesy <a href='http://www.environments.aq/'>environments.aq</a>") %>%
    addGraticule()
mps
\end{verbatim}

\subsection{Examples}\label{examples}

This exmple demonstrates a simple workflow of applying the basic RTools
for data retrieval, cleaning, analysis and visulisation. Users are
encouraged to explore more themselves and apply appropriately.

\begin{verbatim}
install.packages(c("rgbif", "tidyverse", "sp", "raster"))
library(rgbif)
library(tidyverse)
library(sp)
library(mapr)
library(raster)
\end{verbatim}

\subsubsection{occ\_search()}\label{occ_search}

Here we use
\href{\%22https://www.gbif.org/dataset/82d9ff5c-f762-11e1-a439-00145eb45e9a\%22}{Antarctic
Plant Database} as an example. After downloading the entire dataset from
the GBIF, we can start exploring it for a bit. We aim to understand how
each species in this dataset distributes in the interested region
(Antarctica and sub-Antarctica) and across the globe.

\begin{verbatim}
setwd("") #set working directory to where the dataset is saved
occurrence <- read.table("occurrence.txt", header = T, sep = "\t", fill = TRUE, quote = "")
#get an overview of the dataset
head(occurrence)
#find the dimension of the dataset
dim(occurrence) #Here it shows there are 52,181 occurrences
#find how many unique taxon keys there are in the dataset. 
length(unique(occurrence$taxonKey)) #there are 2,653 unique taxon keys
\end{verbatim}

We will retrieve all of the occurrence data of the first 10 taxon keys
by using the GBIF occ\_search() and compute their regional distributions
in Antarctica (60-90\(^\circ\)S), subantarctica (45-60\(^\circ\)S),
south temperate (30-45\(^\circ\)S), tropics
(30\(^\circ\)N-30\(^\circ\)S), north temperate (30-60\(^\circ\)N) and
the arctic (60-90\(^\circ\)N).

First, we write a function to retrieve all of the occurrence data of the
10 taxon keys via using occ\_search() and compute their regional
distributions

\begin{verbatim}
whereGet <- function(taxonKey) {
  dd <- occ_search(taxonKey, return = "data")
  lats <- na.omit(dd$decimalLatitude) #discard occurrences without latitudes
  Antarctic <- subset(lats, lats >= -90 & lats <= -60)
  Subantarctic <- subset(lats, lats > -60 & lats <= -45)
  South_temperate <- subset(lats, lats > -45 & lats <= -30)
  Tropical <- subset(lats, lats > -30 & lats < 30)
  North_temperate <- subset(lats, lats >= 30 & lats < 60)
  Arctic <- subset(lats, lats >= 60 & lats <= 90)

  if(length(lats) != 0) {
    prop_antarctic <- round(length(Antarctic)/length(lats), digits = 3)
    prop_subantarctic <- round(length(Subantarctic)/length(lats), digits = 3)
    prop_south_temperate <- round(length(South_temperate)/length(lats), digits = 3)
    prop_tropical <- round(length(Tropical)/length(lats), digits = 3)
    prop_north_temperate <- round(length(North_temperate)/length(lats), digits = 3)
    prop_arctic <- round(length(Arctic)/length(lats), digits = 3)
    return(c(prop_antarctic = prop_antarctic, prop_subantarctic = prop_subantarctic, prop_south_temperate = prop_south_temperate, prop_tropical = prop_tropical, prop_north_temperate = prop_north_temperate, prop_arctic = prop_arctic))

  } else {
    return(c(prop_antarctic = NA, prop_subantarctic = NA, prop_south_temperate = NA, prop_tropical = NA, prop_north_temperate = NA, prop_arctic = NA))
  }

}
\end{verbatim}

collect the first taxon keys in the dataset

\begin{verbatim}
taxonKeys <- occurrence$taxonKey[1:10] 
props2 <- do.call(rbind, props)
head(props2)
\end{verbatim}

\begin{verbatim}
First, retrieve the occurrence data of krills within the given geometric range. In this example, we set up a limit of 5000 occurrences. This may take a while.
occ_Actinia <- occ_search(scientificName = "Actinia", return = "data", limit = 5000, hasCoordinate = T)
occ_Actinia <- data.frame(occ_Actinia)
#have a view of the data columns
colnames(occ_Actinia)

#You can save the dataset in your favoriate directory.
write.csv(occ_Actinia, ".../occ_Actinia.csv", row.names = FALSE)

#LongLat = CRS("+proj=longlat +ellps=WGS84 +datum=WGS84")
map_leaflet(occ_krills, "decimalLongitude", "decimalLatitude", size=1, color="blue")
xy <- occ_krills[c("decimalLongitude","decimalLatitude")]
#download an environmental layer
env <- getData('worldclim', var='bio', res=10) 
title(main = bquote(italic(.("Euphausia superba")) ~occurrences~on~Annual~mean~temperature~'(dCx10)'))
points(xy, col='blue', pch=20)
\end{verbatim}

\begin{verbatim}
plot(env, 1, main=NULL, axes=FALSE)
\end{verbatim}

the use of R package antanym

\begin{verbatim}
library(antanym)
my_longitude <- c(-180, -120)
my_latitude <- c(-90, -60)
this_names <- an_filter(g, extent = c(my_longitude, my_latitude))
\end{verbatim}

\subsection{ropensci}\label{ropensci}

In addition to the above mentioned R packages and tools,
\href{https://ropensci.org/}{ropensci} is an online forum open to all
interested programmers and scientists for R package development. Users
can find an amazing array of interesting tools and contribute to the
developent of all R packages.

This is an \href{http://rmarkdown.rstudio.com}{R Markdown} Notebook.
When you execute code within the notebook, the results appear beneath
the code.

Try executing this chunk by clicking the \emph{Run} button within the
chunk or by placing your cursor inside it and pressing
\emph{Cmd+Shift+Enter}. `

Add a new chunk by clicking the \emph{Insert Chunk} button on the
toolbar or by pressing \emph{Cmd+Option+I}.

When you save the notebook, an HTML file containing the code and output
will be saved alongside it (click the \emph{Preview} button or press
\emph{Cmd+Shift+K} to preview the HTML file).

The preview shows you a rendered HTML copy of the contents of the
editor. Consequently, unlike \emph{Knit}, \emph{Preview} does not run
any R code chunks. Instead, the output of the chunk when it was last run
in the editor is displayed.


\end{document}
